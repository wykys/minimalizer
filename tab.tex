\documentclass[a3paper]{article}
\usepackage[utf8]{inputenc}

\usepackage[total={30cm,25cm}, top=1cm, left=0.5cm, includefoot]{geometry}
\begin{document}
\section{Karnaughova mapa (modificated by wykys)}
\subsection{rozložení proměnných}
\begin{tabular}[c]{| c | c | c | c | c | c | c | c |}

\hline
[]&['A']&['C']&['C', 'A']&['E']&['E', 'A']&['E', 'C']&['E', 'C', 'A']\\
\hline
['B']&['B', 'A']&['C', 'B']&['C', 'B', 'A']&['E', 'B']&['E', 'B', 'A']&['E', 'C', 'B']&['E', 'C', 'B', 'A']\\
\hline
['D']&['D', 'A']&['D', 'C']&['D', 'C', 'A']&['E', 'D']&['E', 'D', 'A']&['E', 'D', 'C']&['E', 'D', 'C', 'A']\\
\hline
['D', 'B']&['D', 'B', 'A']&['D', 'C', 'B']&['D', 'C', 'B', 'A']&['E', 'D', 'B']&['E', 'D', 'B', 'A']&['E', 'D', 'C', 'B']&['E', 'D', 'C', 'B', 'A']\\
\hline
\end{tabular}
\newline
\newline
\newline
\begin{tabular}[c]{| c | c | c | c | c | c | c | c |}

\hline
0&1&4&5&16&17&20&21\\
\hline
2&3&6&7&18&19&22&23\\
\hline
8&9&12&13&24&25&28&29\\
\hline
10&11&14&15&26&27&30&31\\
\hline
\end{tabular}
\newline
\newline
\newline
\begin{tabular}[c]{| c | c | c | c | c | c | c | c |}

\hline
1&1&0&0&0&1&0&0\\
\hline
1&0&1&1&1&0&1&1\\
\hline
1&1&1&0&0&1&1&0\\
\hline
1&0&1&1&1&0&1&1\\
\hline
\end{tabular}
\newline\newline
\verb|x_size|: 8
\newline
\verb|y_size|: 4
\end{document}
